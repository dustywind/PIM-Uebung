
\chapter{Aufgabenblatt 07}

\section{Datenmodellierung mit XML}
Bestellungen der Mensa sollen innerhalb des E-Procurement-Systems direkt an die Lieferanten weitergeleitet werden.
Die Lieferanten haben dazu jeweils definierte Auftragsschnittellen, die in XML definiert sind.\\

\subsection{Aufgabe 1: XML-Definition}
Bitte erstellen Sie eine XML-Definition (DTD), mit der Bestellung an einen Lieferangen "ubergeben werden und speichern Sie dieses als "`Order.dtd"' ab.
Bitte ber"ucksichtigen Sie dabei die folgenden Anforderungen:
\begin{itemize}
    \item Tag-Namen sind grunds"atzlich in Englisch zu erstellen.
    Die Bestellung (Order) ist das Root-Element.
    \item Jede Bestellung enth"alt Name (SupplierName) und Id (SupplierID) des Lieferanten sowie das Auftragsdatum (OrderDate) und die Auftragsnummer (OrderNumber) als Elemente.
    Eine Bestellung besteht zudem aus mindestens einem Bestellposten (Item).
    \item Jeder Bestellposten bezieht sich auf ein bestimmtes Lebensmittel.
    F"ur das Lebensmittel sollen neben Produktnummer (ItemNumber) sowie der Preis (Price) als Elemente mit angegeben werden.
    Optional ist angegeben, wievielKalorien (Calories) das Nahrungsmittel pro 100g hat.
    \item Der Standard-Lieferant der Mensa, "`Bauernhof Meier"' wird fest im System hinterlegt.
    Bei Eingabe des K"urzels "`BM"' soll dieses automatisch durch die Zeichenkette "`Bauernhof Meier Nuernberg"' ersetzt werden.
\end{itemize}

\noindent
Ob Ihre DTD valide ist k"onnen sie auf folgender Seite "uberpr"ufen:\\
\url{http://www.validome.org/grammar/validate/}\\


\lstinputlisting[style=customXML]{./inc/aufgabe07/order.dtd}

\subsection{Aufgabe 2: XML-Daten}
Bitte erstellen Sie auf Basis der Definition eie XML-Datei (Order.xml), die die DTD referenziert und die mindestens vier Bestellposten enth"alt.
Stellen Sie bitte sicher, dass die erstellte XML Datei sowohl g"ultig als auch wohlgeformt ist.\\

\lstinputlisting[style=customXML]{./inc/aufgabe07/order.xml}

\section{XML/KML und Mashup}
Als besonderes Extra m"ochten Sie in das Mensa Modul Ihres mobilen Universit"atsportals eine Google Maps Karte einbauen.//

\subsection{Aufgabe 1: Statische Positionsbeschreibung}
Sie m"ochten die Positionen der einzelnen Mensen auf einer Google Maps Karte in der Satellitendarstellung anzeigen lassen.\\
Erstellen Sie dazu eine KML Datei und f"ugen Sie die Positionen der folgenden Mensen als einfache Lokation hinzu:
\begin{itemize}
    \item Mensa Langemarckplatz, Langemarckplatz 4, 91054 Erlangen
    \item S"udmensa, Erwin-Rommel-Stra"se 60, 91058 Erlangen
    \item Mensa Insel Sch"utt, Andreij-Sacharow-Platz 1, 90403 N"urnberg
    \item Mensa Regensburger Stra|se, Regensburger Str. 160, 90478 N"urnberg
    \item Ausgabemensa St. Paul, Dutzendteichstra"se 24, 90478 N"urnberg
    \item Mensateria Ohm, Wollentorstr. 4, 90409 N"urnberg
\end{itemize}
Iin der Beschreibung der Positionen sollen sowohl der name der Mensa als auch die vollst"andige Adresse angezeigt werden.\\
Um die Funktionalit"at der von Ihnen erstellten Datei testen zu k"onnen, kopieren Sie bitte den Quelltext in die Zwischenablage und lassen Sie sich die Karte auf folgender Webseite anzeigen:\\
\url{http://display-kml.appspot.com}

\subsection{Aufgabe 2: Polygone in KML}
Zus"atzlich zu den Positionen der einzelnen Mensen m"ochten Sie auch die WISO Fakult"at auf der Karte darstellen.
Sie m"ochten die Einrichtung jedoch nicht nur als einfachen Marker darstellen, sondern das Geb"aude farblich markieren.\\
Erweitern Sie dazu im Text Editor die in Aufgabe 1 erstellte KML Datei so, damit sie folgende Eigenschaften zur Darstellung der farblichen Markierung erf"ullt:
\begin{itemize}
    \item Darstelung eines Polygons
    \item Die "au"sere Kante soll entlang der folgenden Koordination verlaufen:
    \begin{itemize}
        \item 11.084944,49.458151,100
        \item 11.085081,49.457841,100
        \item 11.086463,49.458082,100
        \item 11.086058,49.459037,100
        \item 11.084810,49.458803,100
        \item 11.084875,49.458592,100
        \item 11.085661,49.458713,100
        \item 11.085838,49.458312,100
    \end{itemize}
\end{itemize}
Speichern Sie die Datei unter dem Namen mensafak.kml.\\
Um die Funktionalit"at der von Ihnen erstellten Datei testen zu k"onnen, kopieren Sie bitte den Quelltext in die Zwischenablage und lassen Sie sich die Karte auf folgender Webseite anszeigen:\\ 
\url{http://display-kml.appspot.com}








