

\chapter{Aufgabenblatt 06}

\section{Entwicklung von mobilen Anwendungen}
In der Veranstaltung haben Sie verschiedene M"oglichkeiten kennengelernt, wie mobile Anwendungen entwickelt werden k"onnen.\\
Um die mobile Uni-Applikation zu entwickeln werden Sie um Rat gefragt, welche Form von App sie entwickeln w"urden.

\subsection{Aufgabe 1: Webanwendungen und native Apps}
Das management hat ein attraktives Fremdangebot f"ur die Entwicklung einer nativen iPhone App bekommen.
Betrachten Sie die zu erwartenden Nutzergruppen der Anwendung.
Welche Nachteile k"onnten aus der Entwicklung dieser App entstehen?
Vergleihen Sie die Situation des Marktes f"ur mobile Anwendungen heute mit dem vor 3 Jahren.
Was bedeutet die Ver"anderung f"ur Ihre Entscheidung?


\begin{quote}
    Durch die Entwicklung einer nativen iPhone-App werden nur Nutzer des iPhones angesprochen.\\
    Da jedoch nicht alle Studenten ein iPhone benutzen wird damit nur ein kleiner Teil der Zielgruppe (alle Studenten der FAU) angesprochen.
    Wenn eine native L"osung f"r die MensaApp gew"ahlt werden soll, m"ussen auch native Apps f"ur die restlichen, h"aufig verwendeten Plattformen angeboten werden.
    Das bedeutet, dass eine parallel zur iPhone App je eine weitere App f"ur Android, Windows Phone entwickelt werden muss.
    Dieser Mehraufwand wird sich sowohl zeitlich, als auch monet"ar auswirken.
    Dar"uber hinaus ist es schwerer zu gew"ahrleisten, dass alle Apps die selben Funktionen aufweisen und sich gleich verhalten.
    Ggf. verbieten bestimmte Richtlinien des Plattformanbieters, dass auf allen MensaApps das selbe Design eingesetzt werden kann.

    \noindent
    \emph{Der Folgende Vergleich basiert auf den Prognosen f"ur 2016 und dem Stand von 2013:}\\
    Den Prognosen "uber das Wachstum der verschiedenen Plattformen f"ur mobile Endger"ate kann man entnehmen, dass iOS-Apps in Zukunft zwar immernoch gebraucht werden, der Markt daf"ur aber deutlich kleiner ist als heute.
    Die Android-Plattform hingegen wird den Markt dominieren.
    Um dieser Entwicklung gerecht zu werden sollte nicht in eine native iOS-App investiert werden.\\
    Stattdessen sollte versucht werden alle Plattformen mit einem Produkt zu bedienen.
    Um dieses Ziel zu verwiklichen k"onnen entweder rein Web-basierte, oder hybride  Apps geschrieben werden.
    In beiden F"allen ist man (bei begrenztem Funktionsumfang) weitgehend unabh"angig von der Zielplattform.
\end{quote}

\subsection{Aufgabe 2; Hardware}
Ihr Chef m"ochte sich die M"oglichkeit offen halten in Zukunft Erweiterungen wie Videochats und die intensive Nutzung von Sensoren offen halten.
Was bedeutet as f"ur ihre Entscheidung?
\begin{quote}
    Da Web-basierte Apps keinen Zugriff auf die Hardware des mobilen Ger"ats haben kommen diese nicht mehr in Frage.\\
    Diese M"oglichkeit bieten nur noch native und teilweise hybride Apps.
    Da native Apps jedoch schon ausgeschlo"sen wurden (siehe Aufgabe 1), sollte zu einer hybriden App geraten werden.
\end{quote}

\subsection{Aufgabe 3: Mensa-App}
Bearbeiten Sie Ihre erstellte Webseite so, dass nach der unsortierten Liste ein Button angezeigt wird.
Dieser soll mit den Mitteln von jQuery als Button ausgezeichnet sein.
Den Code finden Sie auf der Seite der jQuery Demo.
Die Beschriftung lautet "`Standorte"', ein Link muss noch nicht hinterlegt werden.

\lstinputlisting[style=customHTML]{./inc/aufgabe06/mensaapp.html}












































