

\chapter{Aufgabenblatt 04}

\section{Datengest"utzte Anwendungen}
In der vergangenen Woche haben Sie das Relationenmodell f"ur die mobile Uni-Applikation erstellt.
Dies liegt mittlerweile in der Datenbank vor.
F"ur die Entwicklung des Mensa-Moduls ben"otigen Sie Ausschnitte aus dieser Datenbank f"ur die Anzeige auf dem mobilen Client.\\
In der Vorlesung haben Sie die Elemente von SQL-Abfragen kennen gelernt.


\subsection{Aufgabe 1: SQL}
Erstellen Sie eine Abgrage, die den Namen und die Bewertung aller Gerichte ausgibt, die in der Zentralmensa in N"urnberg angeboten werden.

\lstset{style=customSQL}
\begin{lstlisting}
SELECT      g.Name, g.Bewertung
FROM        Gericht as g
            JOIN Mensa as m ON g.Mensa = m.Dienststellennummer
WHERE       m.Ort = 'Nuernberg'
            AND m.Name LIKE '%Zentralmensa%'
;
\end{lstlisting}

\subsection{Aufgabe 2: SQL}
Listen Sie den Namen und die Beschreibung aller vegetarischen Gerichte auf, die in den Mensen des Studentenwerks angeboten werden.
Beachten Sie dabei, dass ein Gericht nur dann vegetarisch ist, wenn alle Zutaten vegetarisch sind.\\

\lstset{style=customSQL}
\begin{lstlisting}
-- Achtung!
-- Dieses SQL ist _nur_ auf die in der vorhergehenden
-- Aufgabe definierten Tabellen ausgelegt!
-- Es wurde auf das richtige Ergebnis ueberprueft.

SELECT      RefNum,Name 
FROM        Gericht
WHERE       Gericht.RefNum NOT IN
            (
                -- create a table with all meals containing meat
                SELECT      r.GerichtID
                FROM        Rezept AS r
                            LEFT OUTER JOIN 
                            (
                                SELECT  z.BestNum
                                FROM    Zutat AS z
                                WHERE   z.IstVegetarisch = 1
                            ) AS x
                            ON r.ZutatId = x.BestNum
                -- due to the left-outer join all 
                -- meat-containing meals in table r join x
                -- have their column 'BestNum' set to NULL
                WHERE       x.BestNum IS NULL
            )
;
\end{lstlisting}



\noindent
Beispiel-Tabelle (Rezept):\\
\rowcolors{1}{LightGrey}{White}
\begin{tabular}{ l l }
    \rowcolor{LightSlateGray}
    \textbf{GerichtId}  & \textbf{ZutatId}\\
    id(Currywurst)      & id(Wurst)\\
    id(Currywurst)      & id(Curry)\\
    id(Kartoffelpuffer) & id(Kartoffeln)\\
    id(Kartoffelpuffer) & id(Apfelmus)\\
\end{tabular}



