

\chapter{Aufgabenblatt 01}
\section{Lastenheft}
Da auch die Hochschule mit der Zeit gehen und ihre Studenten bestm¨oglich infomrieren will hat der Dekan beschlossen, eine mobile Applikation entwickeln zu lassen.
Als erstes Modul soll ein Mensa-Informations-System erstellt werden.
Umd die Anwendung in das Budget der Universit¨at einplanen zu k"onnen, bittet man Ihre Abteilung um ein Angebot.\\

\noindent
Sie arbeiten als Software Architekt in der IT-Abteilung der Universit¨at.
Ihr Chef bittet Sie, das vom Dekan erstellte Dokument in ein Anforderungsdokument zu "uberf"uhren.

\subsection{Aufgabe 1: Anforderungsanalyse}
Analysieren Sie das Dokument und "uberlegen Sie sich, ob es m"oglich ist, auf Basis dieser Beschreibung ein Angebot zu erstellen:
\begin{enumerate}
    \item Die Anwendung soll einfach und intuitiv zu bedienen sein und der Gro"steil der Studenten soll sie nutzen k"onnen
    \item Es ist wichtig, dass die Software ansprechend gestaltet ist und den Studenten das Leben vereinfacht
    \item Die App soll die jeweiligen Mensa-Standorte anzeigen k"onnen
    \item Es soll m"oglich sein, die Gerichte der Mensa zu bewerten
    \item Es soll m"oglich sein, wichtige Informationen zur Mensa anzeigen zu lassen
\end{enumerate}

\textbf{L"osung:}
\begin {enumerate}
    \item Die Begriffe "`einfach"',"´intuitiv"',"´Gro"steil"' sind nicht genau spezifiziert, somit nicht SMART- konform
    \item Die Begriffe "´ansprechend gestaltet"' und "´das Leben vereinfacht"' sind nicht genau spezifiziert
    \item Diese Anforderung passt so, wie sie ist
    \item Diese Anforderung passt ebenfalls
    \item Der Begriff "´wichtige Informationen"' sollte n"aher spezifiziert sein
\end{enumerate}
    
\subsection{Aufgabe 2: Erstellung Anforderungsdokument}
Um den Dekan bestm"oglich zu unterst"utzen, erstellen Sie einen Vorschlag f"ur ein detailliertes Anforderungsdokument, das als Grundlage f"ur die sp"atere Entwicklung dienen soll. 
Bitte nutzen Sie dazu die vom Dekan gemachten Angaben als Basis.

\textbf{L"osung:}
\begin{enumerate}
    \item Die app soll eine Einstiegsseite mit einem "Uberblick "uber die Funktionen der App bieten 
    \item Es soll eine Karte mit den relevanten Mensa- Standorten eingebunden werden, welche eine Navigationsm"oglichkeit enth"alt 
    \item Es soll ein Mensa- Modul enthalten sein, welches den Speiseplan und Bewertungsfunktionen enth"alt 
    \item Eine Anbindung an das UNIVIS soll vorhanden sein 
    \item Es soll m"oglich sein, den eigenen Vorlesungsplan zu hinterlegen
    \item E- Learning- Inhalte sollen integriert sein 
    \item "Uber eine Facebook- Anbindung sollen Community-Features inkl. Diskussionsforum realisiert werden 
    \item Das Projekt soll durch Werbung finanziert werden 
    \item Eine Anzeige von Veranstaltungen des t"aglichen Studentenlebens soll m"oglich sein
\end{enumerate}




