

\begin{itemize}
    \item foo
    \begin{enumerate}
        \item foo
        \item bar
    \end{enumerate}
    \item bar
\end{itemize}


\chapter{Aufgabenblatt 02}

\section{HTML}
Sie sind weiterhin mit der Erstellung des Mensa Moduls f"ur die mobile Anwendung der Universit"at beauftragt.
Mittlerweise wurde entschieden, die Anwendung in Form einer mobilen Webseite zu implementieren.
Um sich mit HTML vertraut zu machen versuche Sie, die folgende Struktur als HTML Seite umzusetzen.

\subsection{Erstellung eines HTML-Dockuments}
Erstellen Sie ein neues HTML Dokument mit dem nahem "`MensaApp.html"':
\begin{itemize}
    \item Seitentitel: "`MensaApp"'
    \item "Uberschrift $h1$: "`MensaApp der FAU""
    \item Einf"uhrender Text: "`Herzlich Willkommen auf der STartseite"' in fetter Schrift
    \item Freitext in neuem Absatz: "`Als Student haben Sie die M"oglichkeit an folgenden Mensen zu speisen:"'
    \item Ungeordnete Liste mit folgenden Eintr"agen:
    \begin{itemize}
        \item Mensa Langemarckplatz
        \item S"udmensa
        \item Mensa Insel Sch"utt
        \item Mensa Regensburger Stra"se
        \item Ausgabemensa St. Paul
        \item Mensateria Ohm
    \end{itemize}
    \item Die einzelnen Eintr"age sollen auf die jeweiligen Links zu den entsprechenden Speisepl"anen verweisen, welche hier zu finden sind: \url{http://www.studentenwerk.uni-erlangen.de/verpflegung/de/speiseplaene.shtml}
    \item Pr"ufen Sie das HTML Dokument, indem Sie es sich im Browser anzeigen lassen
\end{itemize}

\lstset{style=customHTML}
\lstinputlisting[style=customHTML]{./inc/aufgabe02/mensaapp.html}

